% In this section, we describe the scientific papers similar to our experiment, both in terms of goal and methodology. 

% One paragraph for each paper (we expect about 5-8 papers to be discussed). Each paragraph contains: (i) a brief description of the related paper and (ii) a black-on-white description about how your experiment differs from the related paper.

\section{Related Work}\label{sec:related}

% An Extensible Approach for Taming the Challenges of JavaScript Dead Code Elimination
% http://www.ivanomalavolta.com/files/papers/SANER_2018.pdf
% MARC
To the best of our knowledge, this paper is the first one to analyze and compare the energy consumption of mobile web pages using (1) the full \acrfull{js} code base, and (2) the de-cluttered \acrshort{js} code base. Most of the research up until now focused on improving and analyzing the performance increase with relation to the \acrfull{plt}. 
First we describe the work of Obbink et al.~\cite{Lacuna} who introduced a tool called \textit{Lacuna}. \textit{Lacuna} automatically eliminates dead \acrshort{js} code from web apps, which in effect improves the \acrshort{plt} and energy consumption by reducing the required network and browser interpretation time. Obbink et al.~\cite{Lacuna} apply different analysis techniques to discover and eliminate dead code from the code base without changing the features of the web app. After testing \textit{Lacuna} on 29 web apps, the results showed that \textit{Lacuna} could be integrated into real-world build systems due to its reasonable execution time. 

%JSCleaner
Second, Chaqfeh et al.~\cite{chaqfeh2020jscleaner} present a way to automatically de-clutter \acrshort{js} code using a tool called \textit{JSCleaner}. The proposed JS de-cluttering tool helps to reduce PLT in web pages without significantly jeopardizing its content and functionality. It follows a rule-based classification algorithm that divides JS into three classes: non-critical, replaceable, and critical. It then deletes non-critical JS from the web pages, interprets replaceable JS components with their corresponding HTML code, and maintains critical JS without any modification. The paper then quantitatively analyzed 500 popular web pages to achieve a 30\% reduction in \acrshort{plt} and 50\% reduction in the number of requests and web page size using the \textit{JSCleaner} tool. For qualitative evaluation, the authors recruited 103 users to analyze the difference between original and de-cluttered JS web pages. The de-cluttered web pages maintained the appearance of the actual web pages with a median value of over 93\%. Both papers \cite{Lacuna, chaqfeh2020jscleaner} introduce a different way to improve the \acrshort{plt}. However, they do not analyze the change in energy consumption of the original versus the cleaned code base. 

% MARC
Other studies perform a similar research methodology, only to accomplish a different goal. For example, Malavolta et al.~\cite{Evaluating_Caching} performed an experiment to test the performance and energy efficiency of \acrshort{pwas} while tuning the browser's cache behavior. The study involved measuring the energy consumption in joules and the \acrshort{plt} in milliseconds during the initial load of nine \acrshort{pwas} on a single Android device running in the Firefox browser. The authors concluded that the current results do not show evidence that adding caching to a website improves its energy efficiency. Despite the goal being different from the one presented in our current paper, the experimental setup and execution are similar to the ones presented in the current research.

Another study that has a similar research methodology is the work of Chan-Jong-Chu~\cite{chan2020investigating}. The authors conducted an empirical experiment using mobile web apps with the aim to find a correlation between performance and energy consumption. The study has a clear research setup in which the correlation is proved using statistical tests together with a thorough description of the experiment execution. Hence, the study is suitable for replication and therefore has high scientific validity. We provide a similar setup to ensure that the measured difference in energy consumption with and without \textit{JSCleaner} is statistically significant and the study can be precisely replicated.

% Who Killed My Battery:
% Analyzing Mobile Browser Energy Consumption
% http://crypto.stanford.edu/~dabo/pubs/papers/browserpower.pdf
Lastly, Thiagarajan et al.~\cite{thiagarajan2012battery} studied the energy consumption of downloading and rendering web pages on mobile devices by comparing the different energy usage (in joules) of web elements. This is relevant to our research as we aim to understand the energy consumption of a specific modification in a mobile web page. Thiagarajan et al.~\cite{thiagarajan2012battery} aim to advise developers in designing energy-efficient mobile web sites. They concluded that the usage of \acrshort{js} in mobile web pages has a negative effect on energy consumption compared to HTML elements. This is relevant to our study as we measure (a potential) energy gain by eliminating non-critical \acrshort{js} code or replacing the \acrshort{js} elements with HTML code. The research of Thiagarajan et al.~\cite{thiagarajan2012battery} differs from our study as they used dedicated hardware to measure the energy consumption whereas we measured the energy consumption using Android Runner Software \cite{malavolta2020runner}.



%\textcolor{red}{Page limit: 1}

% POSSIBLE
% Performance Issues and Optimizations in JavaScript: An Empirical Study
% https://dl.acm.org/doi/pdf/10.1145/2884781.2884829

% (Completely different subject - MAYBE similar method) https://www.usenix.org/legacy/events/hotcloud10/tech/full_papers/Miettinen.pdf

% BAD
% Energy Efficiency across Programming Languages
% https://repositorio.inesctec.pt/bitstream/123456789/5492/1/P-00N-4CX.pdf

% Energy consumption of mobile offloading for JavaScript applications
% https://ieeexplore.ieee.org/abstract/document/7163769

% (Too old) Analysis of the Energy Consumption of JavaScript Based Mobile Web Applications
% https://www.researchgate.net/profile/Antti_Miettinen/publication/220862053_Analysis_of_the_Energy_Consumption_of_JavaScript_Based_Mobile_Web_Applications/links/0912f4ff7565f6c9b4000000.pdf