\section{Introduction}
% This document represents a template of the final experiment report structure for the course \textit{Green Lab} at the Vrije Universiteit Amsterdam \cite{greenlab}.

% The experiment is conducted according to the guidelines by Wohlin and colleagues \cite{wohlin12}.

% The total length of this document must not exceed 15 pages, including references, appendixes, \etc

% In this section you have to describe (i) the domain (\eg mobile apps and their market) and the technologies relevant for understanding the rest of the document, (ii) the main motivation behind your experiment (the problem, here you can show examples via apps/tools screenshots, snippets of source code, \etc), (iii) what your experiment is about (hint of the solution), and (iii) what the developers will learn from the results of your experiment.  

%%% ----- MOBILE TRAFFIC ----- %%%
CISCO predicted global internet traffic in 2020 to be equivalent to 95x the volume of the entire global internet in 2005 \cite{ciscoforecast}. Since the end of the second quarter of 2020, mobile devices (excluding tablets) have been a major contributor to global internet traffic. It accounts for 51.53\% of global website traffic and has an ascending trend. Corresponding to the increased traffic, the complexity of modern web pages has considerably advanced in the previous decade, leading to bigger web pages that are computationally intense for mobile devices. 

%%% ----- JS COMPLEXITY ----- %%%
The extended complexity in websites can be mainly attributed to the inception of web and \acrfull{js} frameworks (such as jQuery, Angular, React) \cite{persson2020javascript} which provide swift solutions for web development with function-specific modules and libraries that can fast pace the programming process. The debut of Single Page Applications (SPAs) further paved the way for more frameworks and standardization of web development. Nevertheless, this also led to \acrshort{js} files in websites being bloated with unused and unnecessary functions from the frameworks and ergo higher computational complexity.

%%% ----- MOBILE TRAFFIC & JS COMPLEXITY EFFECTS -> ENERGY & PLT ----- %%%
A computationally intensive website leads to higher energy consumption as well as increased \acrfull{plt}. It has been shown that web pages that have a \acrshort{plt} higher than 3 seconds, have a probability of losing 32\% of its visitors \cite{googlemobbench}. 
Even if the energy usage of a single smartphone may be considered modest, the overall energy footprint left by mobile devices is far higher \cite{timeenergy}. This was further extended by top-end smartphones and 4G network \cite{chen2014demystifying}. Google handled this by using lighter variants of web pages through Accelerated Mobile Pages (AMP). Another approach for reducing \acrshort{plt} commonly used by web developers is concatenation, minification, and uglification of different \acrshort{js} files in the web page. Even though this approach provides a slightly faster \acrshort{plt} because of smaller file size \cite{improvewebspeed}, \acrshort{js} computations in the browser remain the same. Furthermore, social media websites such as Facebook offers a lighter version of their service (Facebook Lite) which is better suited for low-power devices and more energy efficient.

%%% ----- RELEVANCE & SETUP OUR STUDY ----- %%%
% Our energy analysis
% its relevance
% importance
% motivation for the experiment
Another method to reduce PLT by reducing the amount of \acrshort{js} code is introduced by Chaqfeh et al.~\cite{chaqfeh2020jscleaner}. Chaqfeh et al.~\cite{chaqfeh2020jscleaner} introduced \textit{JSCleaner}: a classification algorithm that classifies critical, replaceable, and non-critical \acrshort{js} code. Accordingly, the replaceable \acrshort{js} code is replaced with HTML counterparts, and non-critical \acrshort{js} code is eliminated. This process is referred to as `de-cluttering'. The authors studied the relation between the de-cluttered web pages and the \acrshort{plt}. They concluded that de-cluttering achieves a 30\% reduction in \acrshort{plt}. A different study, conducted by Thiagarajan et al.~\cite{thiagarajan2012battery} showed that \acrshort{js} code has a relatively high energy consumption compared to other web elements. Since energy-efficiency contributes to the user experience and impact of increasingly popular mobile web pages, we aim to contribute by investigating the impact of eliminating \acrshort{js} code on energy consumption. Our research focuses on the relation between de-cluttered \acrshort{js} code using \textit{JSCleaner} on mobile web pages and its impact on energy consumption. Hence, our research question is: ``\textit{What is the impact of de-cluttering JavaScript code in mobile web pages using \textit{JScleaner} with respect to the energy consumption?}''.

%%% ----- EXPERIMENT DETAILS & PAPER STRUCTURE ----- %%%
% how will we answer the research question?

The remainder of this paper is structured as follows. Section 2 provides an insight into related work, and how this paper is compared to these related works. Next, section 3 defines the experiment using the GQM method. 

%\textcolor{red}{Page limit: 2}