% In this section, we describe the scientific papers similar to our experiment, both in terms of goal and methodology. 

% One paragraph for each paper (we expect about 5-8 papers to be discussed). Each paragraph contains: (i) a brief description of the related paper and (ii) a black-on-white description about how your experiment differs from the related paper.

\section{Related Work}\label{sec:related}

% An Extensible Approach for Taming the Challenges of JavaScript Dead Code Elimination
% http://www.ivanomalavolta.com/files/papers/SANER_2018.pdf
To the best of our knowledge, this paper is the first one to analyze and compare the energy consumption of mobile web apps using (1) the cluttered \acrshort{js} code base, and (2) the de-cluttered \acrshort{js} code base. Most of the research up until now focused on improving and analyzing the performance increase with relation to the \acrshort{plt}. First, we describe the work of Obbink et al.~\cite{Lacuna}, who introduced a tool called \textit{Lacuna}. Lacuna automatically eliminates dead \acrshort{js} code from mobile web apps, which in effect improves the \acrshort{plt} and energy consumption by reducing the required network and browser interpretation time. Obbink et al.~\cite{Lacuna} apply different analysis techniques to discover and eliminate dead code from the code base without changing the features of the mobile web app. After testing Lacuna on 29 mobile web apps, the results showed that Lacuna could be integrated into real-world build systems due to its reasonable execution time. 

%JSCleaner
Second, Chaqfeh et al.~\cite{chaqfeh2020jscleaner} present a way to automatically de-clutter \acrshort{js} code using a tool called JSCleaner. The proposed JS de-cluttering tool helps to reduce PLT in mobile web apps without significantly jeopardizing its content and functionality. It follows a rule-based classification algorithm that divides JS code into three classes: non-critical, replaceable, and critical. It then deletes non-critical JS code from the mobile web apps, interprets replaceable JS components with their corresponding HTML code, and maintains critical JS code without any modification. The study then quantitatively analyzed 500 popular mobile web apps and reports an empirical evaluation in which a 30\% reduction in \acrshort{plt} and a 50\% reduction in the number of requests and size using the JSCleaner tool is presented. A qualitative evaluation followed in which the authors recruited 103 users to analyze the difference in functionality between cluttered and de-cluttered mobile web apps. The de-cluttered mobile web apps maintained the appearance of the cluttered mobile web apps with a median value of over 93\%. Both papers \cite{Lacuna, chaqfeh2020jscleaner} introduce a different way to improve the \acrshort{plt}. However, they do not analyze the change in energy consumption of the cluttered versus the de-cluttered code base. 

Other studies perform a similar research methodology, only to accomplish a different goal. For example, Malavolta et al.~\cite{Evaluating_Caching} performed an experiment to test the performance and energy efficiency of \acrfull{pwas} while tuning the browser's cache behavior. The study involved measuring the energy consumption in joules and the \acrshort{plt} in milliseconds during the initial load of nine \acrshort{pwas} on a single Android device running in the Firefox browser. The authors concluded that the current results do not show evidence that adding caching to a website improves its energy efficiency. Despite the goal being different from the one presented in our study, the experimental setup and execution are similar to the ones presented in our study.

Another study that has a similar research methodology is the work of Chan-Jong-Chu~\cite{chan2020investigating}. The authors conducted an empirical experiment using mobile web apps with the aim to find a correlation between performance and energy consumption. The study has a clear research setup in which the correlation is proved using statistical tests together with a thorough description of the experiment execution. We provide a similar setup to ensure that the measured difference in energy consumption of cluttered and de-cluttered mobile web apps is statistically significant and the study can be precisely replicated.

% Who Killed My Battery:
% Analyzing Mobile Browser Energy Consumption
% http://crypto.stanford.edu/~dabo/pubs/papers/browserpower.pdf
Lastly, Thiagarajan et al.~\cite{thiagarajan2012battery} studied the energy consumption of downloading and rendering mobile web apps on mobile devices by comparing the different energy usage (in joules) of web elements. This is relevant to our research as we aim to understand the energy consumption of a specific modification in a mobile web app. Thiagarajan et al.~\cite{thiagarajan2012battery} aim to advise developers in designing energy-efficient mobile web apps. They concluded that the usage of \acrshort{js} code in mobile web apps has a negative effect on energy consumption compared to HTML elements. This is relevant to our study as we measure (a potential) energy gain by eliminating non-critical \acrshort{js} code or replacing the \acrshort{js} elements with HTML code. The main difference between the study of Thiagarajan et al.~\cite{thiagarajan2012battery} and our study is that the work of Thiagarajan et al. is more descriptive (they are trying to understand a phenomenon), whereas we aim at discovering a specific relationship (i.e., the one between being cluttered and consuming different amounts of energy).

%\textcolor{red}{Page limit: 1}