\documentclass[sigconf,review]{acmart}

\usepackage{amsmath,amssymb,amsfonts,latexsym}
\usepackage{enumerate}
\usepackage{xspace}
\usepackage{epsf,picinpar}
\usepackage{varioref}
\usepackage{colortbl,multirow,hhline}
\usepackage{listings}
\usepackage{amssymb}
\usepackage{colortbl,multirow,hhline}
\usepackage{algorithmic}
\usepackage{algorithm}
\usepackage{caption}
\usepackage[normalem]{ulem}
\usepackage{xcolor}
\usepackage{pifont}
\usepackage{xcolor,colortbl}
\usepackage{url}
\usepackage{balance}
\usepackage{graphicx, subfigure}
\usepackage{longtable}
\usepackage{lscape}
\usepackage{multirow}
\usepackage{listings}
\usepackage{framed}
\usepackage{morefloats}
\usepackage[T1]{fontenc}
\usepackage{array}
\usepackage{pdfpages}
\usepackage{fancybox}
\usepackage{amsmath}
\usepackage{flushend}
\usepackage{booktabs}
\usepackage{enumitem}
\usepackage[acronym]{glossaries}
\usepackage{wrapfig}

% \usepackage[table]{xcolor}


\renewcommand{\ttdefault}{cmr}

%\newcommand{\limit}[1]{\textcolor{red}{\noindent \ding{46}~Page limit:~#1}\\}
\newcommand{\todo}[1]{\textcolor{red}{\ding{46}~#1}} 
\newcommand{\ie}{\emph{i.e.,}\xspace}
\newcommand{\eg}{\emph{e.g.,}\xspace}
\newcommand{\etc}{etc.\xspace}
\newcommand{\etal}{\emph{et~al.}\xspace} 

%%% ----- ACRONYMS ----- %%%
\makeglossaries
\newacronym{plt}{PLT}{Page Load Time}
\newacronym{js}{JS}{JavaScript}
\newacronym{pwas}{PWAs}{Progressive Web Applications}
\newacronym{pwa}{PWA}{Progressive Web Application}

\copyrightyear{2020}
\acmYear{2020}
\setcopyright{acmcopyright}
\acmConference[Green Lab 2020/2021]{Green Lab 2020/2021 - Vrije Universiteit Amsterdam}{September--October, 2020}{Amsterdam, The Netherlands}
\acmBooktitle{Green Lab 2020/2021 - Vrije Universiteit Amsterdam, September--October, 2020, Amsterdam (The Netherlands)}
    
\begin{document}

\title{
	The Impact of Cluttered JavaScript Code in Mobile Web Apps with respect to Energy Consumption
}


\author{Geoffrey van Driessel}
\affiliation{%
 \institution{2639310 \\ VU Amsterdam}
} \email{g.r.van.driessel@student.vu.nl}

\author{Abijith Radhakrishnan}
\affiliation{%
 \institution{2667575 \\ VU Amsterdam}
} \email{a.radhakrishnan@student.vu.nl}

\author{Sophie Vos}
\affiliation{%
 \institution{2551583 \\ VU Amsterdam}
} \email{s.o.vos@student.vu.nl}

\author{Marc Wiggerman}
\affiliation{%
 \institution{2653796 \\ VU Amsterdam}
} \email{m.g.wiggerman@student.vu.nl}

% \author{Name Surname}
% \affiliation{%
%  \institution{Student Number \\ VU Amsterdam}
% } \email{xyz@student.vu.nl}

\begin{abstract}
\noindent \textit{Context}. 
Mobile phones are a major contributor to global internet traffic. Although, the energy usage of a single smartphone may be considered modest, the overall energy footprint left by mobile devices is far higher. Thiagarajan et al.~\cite{thiagarajan2012battery}, showed that JavaScript (JS) code mainly contributes to the energy consumption relative to other web elements. Chaqfeh et al.\cite{chaqfeh2020jscleaner}~introduced a tool, called \textit{JSCleaner}, that de-clutters mobile web apps by removing non-essential JS code. They showed that this positively influences the Page Load Time. 

\noindent \textit{Goal}. 
Since energy-efficiency contributes to the user experience and the environmental impact of increasingly popular mobile web apps, we aim to contribute by investigating the impact of cluttered JS code on energy consumption. The goal of our study is: ``Analyze cluttered JS code for the purpose of evaluation with respect to the energy consumption from the point of view of software developers in the context of mobile web apps".

\noindent \textit{Method}. 
To accomplish this goal, we conducted an empirical experiment to measure and compare the energy consumption of cluttered and de-cluttered mobile web apps. We randomly selected 27 diverse, currently operational mobile web apps as subjects. Furthermore, we used Android Runner and the Batterystats profiler to measure the energy consumption. 

\noindent \textit{Results}. We tested the energy consumption for small, medium, and large mobile web apps. Our results showed that cluttered mobile web apps do not influence the energy consumption. 

\noindent \textit{Conclusions}. Hence, based on this experiment, we do not recommend developers who aim to reduce the energy consumption of their mobile web app to apply a de-cluttering engine. It should be noted that we tested the difference in energy consumption using a relatively small dataset and using a single de-cluttering engine. Therefore, further research is required to support this claim.
\end{abstract}

\maketitle

\section{Introduction}
% This document represents a template of the final experiment report structure for the course \textit{Green Lab} at the Vrije Universiteit Amsterdam \cite{greenlab}.

% The experiment is conducted according to the guidelines by Wohlin and colleagues \cite{wohlin12}.

% The total length of this document must not exceed 15 pages, including references, appendixes, \etc

% In this section you have to describe (i) the domain (\eg mobile apps and their market) and the technologies relevant for understanding the rest of the document, (ii) the main motivation behind your experiment (the problem, here you can show examples via apps/tools screenshots, snippets of source code, \etc), (iii) what your experiment is about (hint of the solution), and (iii) what the developers will learn from the results of your experiment.  

%%% ----- MOBILE TRAFFIC ----- %%%
CISCO predicted global internet traffic in 2020 to be equivalent to 95x the volume of the entire global internet in 2005 \cite{ciscoforecast}. Since the end of the second quarter of 2020, mobile devices (excluding tablets) have been a major contributor to global internet traffic. It accounts for 51.53\% of global website traffic and has an ascending trend. Corresponding to the increased traffic, the complexity of modern web pages has considerably advanced in the previous decade, leading to bigger web pages that are computationally intense for mobile devices. 

%%% ----- JS COMPLEXITY ----- %%%
The extended complexity in websites can be mainly attributed to the inception of web and \acrfull{js} frameworks (such as jQuery, Angular, React) \cite{persson2020javascript} which provide swift solutions for web development with function-specific modules and libraries that can fast pace the programming process. The debut of Single Page Applications (SPAs) further paved the way for more frameworks and standardization of web development. Nevertheless, this also led to \acrshort{js} files in websites being bloated with unused and unnecessary functions from the frameworks and ergo higher computational complexity.

%%% ----- MOBILE TRAFFIC & JS COMPLEXITY EFFECTS -> ENERGY & PLT ----- %%%
A computationally intensive website leads to higher energy consumption as well as increased \acrfull{plt}. It has been shown that web pages that have a \acrshort{plt} higher than 3 seconds, have a probability of losing 32\% of its visitors \cite{googlemobbench}. 
Even if the energy usage of a single smartphone may be considered modest, the overall energy footprint left by mobile devices is far higher \cite{timeenergy}. This was further extended by top-end smartphones and 4G network \cite{chen2014demystifying}. Google handled this by using lighter variants of web pages through Accelerated Mobile Pages (AMP). Another approach for reducing \acrshort{plt} commonly used by web developers is concatenation, minification, and uglification of different \acrshort{js} files in the web page. Even though this approach provides a slightly faster \acrshort{plt} because of smaller file size \cite{improvewebspeed}, \acrshort{js} computations in the browser remain the same. Furthermore, social media websites such as Facebook offers a lighter version of their service (Facebook Lite) which is better suited for low-power devices and more energy efficient.

%%% ----- RELEVANCE & SETUP OUR STUDY ----- %%%
% Our energy analysis
% its relevance
% importance
% motivation for the experiment
Another method to reduce PLT by reducing the amount of \acrshort{js} code is introduced by Chaqfeh et al.~\cite{chaqfeh2020jscleaner}. Chaqfeh et al.~\cite{chaqfeh2020jscleaner} introduced \textit{JSCleaner}: a classification algorithm that classifies critical, replaceable, and non-critical \acrshort{js} code. Accordingly, the replaceable \acrshort{js} code is replaced with HTML counterparts, and non-critical \acrshort{js} code is eliminated. This process is referred to as `de-cluttering'. The authors studied the relation between the de-cluttered web pages and the \acrshort{plt}. They concluded that de-cluttering achieves a 30\% reduction in \acrshort{plt}. A different study, conducted by Thiagarajan et al.~\cite{thiagarajan2012battery} showed that \acrshort{js} code has a relatively high energy consumption compared to other web elements. Since energy-efficiency contributes to the user experience and impact of increasingly popular mobile web pages, we aim to contribute by investigating the impact of eliminating \acrshort{js} code on energy consumption. Our research focuses on the relation between de-cluttered \acrshort{js} code using \textit{JSCleaner} on mobile web pages and its impact on energy consumption. Hence, our research question is: ``\textit{What is the impact of de-cluttering JavaScript code in mobile web pages using \textit{JScleaner} with respect to the energy consumption?}''.

%%% ----- EXPERIMENT DETAILS & PAPER STRUCTURE ----- %%%
% how will we answer the research question?

The remainder of this paper is structured as follows. Section 2 provides an insight into related work, and how this paper is compared to these related works. Next, section 3 defines the experiment using the GQM method. 

%\textcolor{red}{Page limit: 2}
% In this section, we describe the scientific papers similar to our experiment, both in terms of goal and methodology. 

% One paragraph for each paper (we expect about 5-8 papers to be discussed). Each paragraph contains: (i) a brief description of the related paper and (ii) a black-on-white description about how your experiment differs from the related paper.

\section{Related Work}\label{sec:related}

% An Extensible Approach for Taming the Challenges of JavaScript Dead Code Elimination
% http://www.ivanomalavolta.com/files/papers/SANER_2018.pdf
To the best of our knowledge, this paper is the first one to analyze and compare the energy consumption of mobile web apps using (1) the cluttered \acrshort{js} code base, and (2) the de-cluttered \acrshort{js} code base. Most of the research up until now focused on improving and analyzing the performance increase with relation to the \acrshort{plt}. First, we describe the work of Obbink et al.~\cite{Lacuna}, who introduced a tool called \textit{Lacuna}. Lacuna automatically eliminates dead \acrshort{js} code from mobile web apps, which in effect improves the \acrshort{plt} and energy consumption by reducing the required network and browser interpretation time. Obbink et al.~\cite{Lacuna} apply different analysis techniques to discover and eliminate dead code from the code base without changing the features of the mobile web app. After testing Lacuna on 29 mobile web apps, the results showed that Lacuna could be integrated into real-world build systems due to its reasonable execution time. 

%JSCleaner
Second, Chaqfeh et al.~\cite{chaqfeh2020jscleaner} present a way to automatically de-clutter \acrshort{js} code using a tool called JSCleaner. The proposed JS de-cluttering tool helps to reduce PLT in mobile web apps without significantly jeopardizing its content and functionality. It follows a rule-based classification algorithm that divides JS code into three classes: non-critical, replaceable, and critical. It then deletes non-critical JS code from the mobile web apps, interprets replaceable JS components with their corresponding HTML code, and maintains critical JS code without any modification. The study then quantitatively analyzed 500 popular mobile web apps and reports an empirical evaluation in which a 30\% reduction in \acrshort{plt} and a 50\% reduction in the number of requests and size using the JSCleaner tool is presented. A qualitative evaluation followed in which the authors recruited 103 users to analyze the difference in functionality between cluttered and de-cluttered mobile web apps. The de-cluttered mobile web apps maintained the appearance of the cluttered mobile web apps with a median value of over 93\%. Both papers \cite{Lacuna, chaqfeh2020jscleaner} introduce a different way to improve the \acrshort{plt}. However, they do not analyze the change in energy consumption of the cluttered versus the de-cluttered code base. 

Other studies perform a similar research methodology, only to accomplish a different goal. For example, Malavolta et al.~\cite{Evaluating_Caching} performed an experiment to test the performance and energy efficiency of \acrfull{pwas} while tuning the browser's cache behavior. The study involved measuring the energy consumption in joules and the \acrshort{plt} in milliseconds during the initial load of nine \acrshort{pwas} on a single Android device running in the Firefox browser. The authors concluded that the current results do not show evidence that adding caching to a website improves its energy efficiency. Despite the goal being different from the one presented in our study, the experimental setup and execution are similar to the ones presented in our study.

Another study that has a similar research methodology is the work of Chan-Jong-Chu~\cite{chan2020investigating}. The authors conducted an empirical experiment using mobile web apps with the aim to find a correlation between performance and energy consumption. The study has a clear research setup in which the correlation is proved using statistical tests together with a thorough description of the experiment execution. We provide a similar setup to ensure that the measured difference in energy consumption of cluttered and de-cluttered mobile web apps is statistically significant and the study can be precisely replicated.

% Who Killed My Battery:
% Analyzing Mobile Browser Energy Consumption
% http://crypto.stanford.edu/~dabo/pubs/papers/browserpower.pdf
Lastly, Thiagarajan et al.~\cite{thiagarajan2012battery} studied the energy consumption of downloading and rendering mobile web apps on mobile devices by comparing the different energy usage (in joules) of web elements. This is relevant to our research as we aim to understand the energy consumption of a specific modification in a mobile web app. Thiagarajan et al.~\cite{thiagarajan2012battery} aim to advise developers in designing energy-efficient mobile web apps. They concluded that the usage of \acrshort{js} code in mobile web apps has a negative effect on energy consumption compared to HTML elements. This is relevant to our study as we measure (a potential) energy gain by eliminating non-critical \acrshort{js} code or replacing the \acrshort{js} elements with HTML code. The main difference between the study of Thiagarajan et al.~\cite{thiagarajan2012battery} and our study is that the work of Thiagarajan et al. is more descriptive (they are trying to understand a phenomenon), whereas we aim at discovering a specific relationship (i.e., the one between being cluttered and consuming different amounts of energy).

%\textcolor{red}{Page limit: 1}
\section{Experiment Definition}

As described in the related work section, \textit{JSCleaner} is able to reduce the size of \acrshort{js} files significantly and, therefore, reduce the \acrshort{plt}. However, it has not been shown that this actually reduces the energy consumption. The \textbf{goal} of our study is: ``\textit{Analyze JSCleaner for the purpose of evaluation with respect to the energy consumption from the point of view of software developers in the context of mobile web pages}". This goal has been established using the GQM method ~\cite{claes2012experimentation}, the steps are presented in Table \ref{tab:gqm}.

\begin{table}[H]
    \caption{GQM}
    \centering
    \begin{tabular}{||c || c||} 
     \hline
     Analyse & JSCleaner \\ 
     \hline
     For the purpose of & evaluating \\
     \hline
     With respect to the & energy consumption \\
     \hline
     From point of view of & developer \\
     \hline
     In the context of & mobile web pages \\
     \hline
    \end{tabular}
\label{tab:gqm}
\end{table}


Following our goal we have identified the main \textbf{research question}: \textbf{[RQ1]} ``\textit{What is the impact of de-cluttering JavaScript code in mobile web pages using \textit{JScleaner} with respect to the energy consumption?}''. We answer this research question by comparing the energy consumption of 500 web pages compared to their de-cluttered version. 

To answer our research question the \textit{energy consumption (joule)} is used as \textbf{metric}. The energy consumption is calculated using the measured \textit{power consumption (joule per second)} times the \textit{\acrshort{plt} (seconds)}.

Figure \ref{fig:gqm} presents a visual representation of the GQM. It hierarchically shows how the goal is obtained using the research question, and how the research question is answered using the metrics.

\begin{figure}[t]
\caption{GQM tree}
\includegraphics[width=8cm]{reportTemplate/figures/GQM.pdf}
\centering
\label{fig:gqm}
\end{figure}

%\textcolor{red}{Page limit: 2}
\section{Experiment Planning}

In this section, we explain how the experiment is laid out. This is done in the following steps: context, subjects, experimental variables, hypothesis, experiment design, and data analysis.

%%% --- Context Selection --- %%%

\subsection{Context Selection}

To characterize the experiments of our study, the four dimension method presented by Wohlin et al.~\cite{wohlin2012experimentation} is used.  

The first dimension regards the choice between on-line and off-line experiments. As the experiments conducted in our study use a local Android device loading pre-developed mobile web apps, the off-line dimension is selected.

The second dimension represents the choice of using students versus professionals as subjects. This dimension is not suitable for our study as we use mobile web apps as subjects. No humans are the subject of the performed experiments.

The third dimension determines if the nature of the experiments are real or toy problems. As the mobile web apps used in this study are collected from real mobile web apps, the experiments and results presented in this paper can be regarded as covering real problems. Moreover, the outcome of our study could also directly impact real problems, i.e. developers could immediately tackle the energy efficiency of their real mobile web apps using a de-cluttering tool.

The fourth and final dimension defines whether the experiments are specific versus general. Our study can be regarded as general as it uses diverse and frequently visited mobile web apps to conduct the experiments. In other words, no specific type of mobile web app is selected.

%%% --- Subjects Selection --- %%%

\subsection{Subjects Selection}\label{subsec:subjects_selection} 

The subject selection is done by a Python script (available through \textcolor{blue}{\url{https://github.com/TheEnergyEngineers/ClutteredJS}}) that reads a CSV file containing $93$ mobile web apps and their corresponding size. The mobile web apps are a \textcolor{blue}{subset of the mobile web apps} used by Chaqfeh et al.~\cite{chaqfeh2020jscleaner} containing $93$ popular mobile web apps. The set is obtained from The Majestic Million\footnote{https://majestic.com/reports/majestic-million} and includes a wide variety of mobile web apps such as: news sites, education, sports, entertainment, travel, and government web apps. This ensures the generalizability of our sample to the real-world population of mobile web apps. The mobile web apps are available on a proxy server provided by Chaqfeh et al. \cite{chaqfeh2020jscleaner}. The proxy server contains both the cluttered mobile web apps and the mobile web apps de-cluttered by JSCleaner. 

To perform the analysis required for the \textcolor{blue}{subject} selection, each mobile web app is downloaded locally using \textit{Resources Saver Extension} for \textit{Google Chrome}\footnote{\url{https://github.com/up209d/ResourcesSaverExt}} after which its size is measured and a \textit{Lines Of Code} analysis is performed using \textit{CLOC}\footnote{\url{https://github.com/AlDanial/cloc}}. \textcolor{blue}{Each page load and download is given one minute for each stage to complete.}

\textcolor{blue}{Before the subjects are selected, some mobile web apps have to be removed from the data set. First, pages with a size of 0 bytes are removed. This size occurs when the page failed to load on the proxy. Second, pages that occur more than once in the dataset are removed. The third group being removed are the pages which \--- according to CLOC \--- do not contain JS, HTML, or CSS code. For all these mobile web apps, CLOC did not report a value for any one of these three programming languages. After visually analyzing the deleted mobile web apps using Google Chrome, we observe that most mobile web apps do not show any visible changes after de-cluttering the mobile web apps. The complete deletion process results in \textcolor{blue}{$74$} mobile web apps that are suitable for analysis.} 

After this cleaning step, the mobile web apps are divided into three groups based on the size of the cluttered mobile web app. These groups are created to control the effect of size on the energy consumption. The groups are created using three quantiles and contain the following size ranges:

\textcolor{blue}{
\begin{tabular}{lll}
Small & 0.0 MB - 4.6 MB & (24 mobile web apps)\\
Medium & 4.7 MB - 7.5 MB & (24 mobile web apps)\\
Large & 8.1 MB - 46 MB & (26 mobile web apps) 
\end{tabular}
}

After this division, the script randomly selects 10 mobile web apps from each group using a pseudo-random number generator\footnote{\url{https://docs.python.org/3/library/random.html}} with a seed of 42. This process completes with a total of 30 mobile web apps (each \textcolor{blue}{mobile web} app is used twice during the experiments, once as the cluttered version and once as the de-cluttered version). The selected mobile web apps and their information is presented in Table \ref{tab:sample_of_subject}.

\textcolor{blue}{From Table \ref{tab:sample_of_subject}, we observe that some de-cluttered mobile web apps are larger compared to the cluttered version. After a visual inspection, we found that most cluttered and de-cluttered mobile web apps do not differ in functionality. Some mobile web apps do show a difference. For example, the cluttered version of `\url{store.steampowered.com}' shows a list of games and their corresponding images. In the de-cluttered version, these games are not loaded. This causes a difference in size between the two versions due to JSCleaner. In the case when the cluttered and de-cluttered versions do not differ in functionality, we often see that the total lines of code (JS, HTML, and CSS) increases in the de-cluttered version compared to the cluttered version. This again could be a side-effect of JSCleaner which alters the code of the cluttered version.}

% 
\begin{table*}[t]
\centering
\tiny
\textcolor{blue}{
\begin{tabular}{|l|ll|l|l|lll|lll|}
\toprule
ID &                  URL & Cluttered Size & De-cluttered Size & Category & OG LOC JS & OG LOC HTML & OG LOC CSS & DC LOC JS & DC LOC HTML & DC LOC CSS \\
\midrule
0  &                 fc2.com &          572KB &           564KB &    Small &      4521 &         326 &       2210 &      4505 &         326 &       2210 \\
1  &            telegram.org &          920KB &           828KB &    Small &      2794 &         321 &       6494 &       550 &         321 &       6494 \\
2  &             www.nih.gov &          1.3MB &           2.1MB &    Small &     22457 &           4 &        555 &     29155 &         866 &      12125 \\
3  &          www.office.com &          1.7MB &            18MB &    Small &     13458 &        1256 &       8119 &    238496 &        1338 &      21820 \\
4  &          duckduckgo.com &          1.8MB &           360KB &    Small &     35433 &         109 &      10031 &         4 &          98 &      10031 \\
5  &              www.rt.com &          3.0MB &           2.9MB &    Small &     38487 &        3425 &      21329 &     36237 &        3425 &      21329 \\
6  &            www.ucsc.edu &          3.3MB &           2.0MB &    Small &     25922 &        4631 &        464 &     11441 &         410 &       4976 \\
7  &        www.colorado.edu &          3.6MB &           264KB &    Small &     36375 &        4635 &        265 &      1630 &           4 &         80 \\
8  &             pixabay.com &          4.0MB &           1.3MB &    Small &     34116 &        5417 &       4724 &      9582 &         786 &       4724 \\
9  &                 utk.edu &          4.2MB &           1.5MB &    Small &     36161 &        5545 &       8506 &     11624 &         921 &       8506 \\\midrule
10 &       www.sindonews.com &          4.7MB &           2.2MB &   Medium &     60680 &        4189 &       4823 &     23064 &         204 &        616 \\
11 &            www.osha.gov &          4.7MB &           4.5MB &   Medium &     34327 &        3144 &      19679 &     32073 &        3145 &      19679 \\
12 &            missouri.edu &          5.6MB &           2.7MB &   Medium &     35247 &        5403 &      15862 &      8301 &         772 &      15862 \\
13 &         www.booking.com &          6.0MB &           4.1MB &   Medium &     61556 &       10902 &      54391 &     35001 &       10396 &      53983 \\
14 &            www.mdpi.com &          6.2MB &           3.5MB &   Medium &     41608 &        9578 &      16062 &     17075 &        4947 &      16062 \\
15 &     www.raspberrypi.org &          6.2MB &           5.9MB &   Medium &     46908 &       16545 &        464 &     44657 &       15820 &        464 \\
16 &       www.homedepot.com &          6.6MB &           5.5MB &   Medium &     16739 &        3067 &          3 &     14715 &        3067 &          3 \\
17 &          ae.godaddy.com &          6.8MB &           6.8MB &   Medium &     92350 &        8111 &      31031 &     92352 &        8112 &      31031 \\
18 &  store.steampowered.com &          7.1MB &           6.9MB &   Medium &     17373 &       13846 &      11567 &     15129 &       13846 &      11567 \\
19 &           www.indeed.ae &          7.3MB &           6.3MB &   Medium &     61220 &        4714 &      20372 &     42071 &        2464 &      19859 \\\midrule
20 &        www.typeform.com &          8.1MB &           3.7MB &    Large &    102582 &        8770 &       1044 &     43485 &        4138 &       1044 \\
21 &    www.researchgate.net &          8.1MB &           4.7MB &    Large &    116086 &        5196 &      27014 &     77066 &         560 &      27014 \\
22 &            www.yelp.com &          8.7MB &           8.4MB &    Large &     56665 &       17865 &      43263 &     47658 &       17865 &      43263 \\
23 &              kakaku.com &          9.0MB &           6.3MB &    Large &    101395 &       13735 &       6548 &     78069 &        9104 &       6548 \\
24 &      steamcommunity.com &          9.1MB &           1.5MB &    Large &     22514 &        1940 &       8093 &     20193 &         538 &       8093 \\
25 &         www.tistory.com &          9.6MB &           9.6MB &    Large &     24184 &         495 &       7023 &     24082 &         495 &       7023 \\
26 &         www.netflix.com &           11MB &            11MB &    Large &      1083 &       23732 &       5781 &      1083 &       23732 &       5781 \\
27 &       www.worldbank.org &           11MB &            12MB &    Large &     60677 &       62704 &      63298 &     57052 &       64838 &      63298 \\
28 &          www.ltn.com.tw &           28MB &            28MB &    Large &    225749 &        6517 &      25981 &    223069 &        6102 &      25981 \\
29 &      www.thetimes.co.uk &           31MB &            26MB &    Large &    354557 &       25924 &        787 &    338192 &       25500 &        768 \\
\bottomrule
\end{tabular}
}

\caption{The subjects used during the experiments (OG = Original, DC = De-Cluttered, LOC = Lines of Code).}
\label{tab:sample_of_subject}
\end{table*}


%%% --- Experimental Variables --- %%%

\subsection{Experimental Variables}

In our study, the \textit{cluttering state} is regarded as the independent variable. The independent variable is both cluttered and de-cluttered once for each selected subject. The \textit{energy consumption (in joules)} is regarded as the dependent variable which is measured in the performed experiments. The energy consumption is calculated by multiplying the \textit{power consumption (in watts)} and the \textit{PLT (in seconds)}.

To reduce the effect of the page size on the results, the size categories introduced in Section \ref{subsec:subjects_selection} are used as a blocking variable. Next, to reduce the effect of co-factors such as the browser type, network type, and device on the results, these co-factors are fixed to a single value. 

%%% --- Experimental Hypotheses --- %%%

\subsection{Experimental Hypotheses}

To test the research question, we have defined the following hypotheses. The null hypothesis states that the average energy consumption for cluttered mobile web apps is equal to the average energy consumption of de-cluttered mobile web apps:
$$H_0: \mu_{E_{cluttered}} = \mu_{E_{de-cluttered}}$$
where $E$ is the measured energy consumption and the subscript specifies the treatment.
The alternative hypothesis states that the average energy consumption for cluttered mobile web apps is not equal to the average energy consumption of de-cluttered mobile web apps:
$$H_a: \mu_{E_{cluttered}} \neq \mu_{E_{de-cluttered}}$$


%%% --- Experiment Design --- %%%

\subsection{Experiment Design}

The experiment is performed according to the following method: for each selected subject (which are described in Section 4.2) we have each treatment applied. This entails that there is a cluttered and de-cluttered version of each subject. In this way, each block of the mobile web app size is automatically balanced (crossover design). Both versions (cluttered and de-cluttered) of the mobile web app are loaded (in random order) on a smartphone to measure the power consumption and PLT. Our experiment design has the structure: 1 factor - 2 treatments \cite{wohlin2012experimentation}. The cluttering state is the factor and \textit{<cluttered, de-cluttered>} are the treatments. The energy consumption is the measured outcome of the experiment. We repeat each trial (a combination of subject and treatment) 15 times to mitigate the effect of measurement errors. To summarize, the energy consumption of each mobile web app is measured 15 times for the cluttered version and 15 times for the de-cluttered version. Hence, as there are 10 subjects in each of the three blocks \textit{<small, medium, large>}, there are:
3 (blocks) $x$ (10 (cluttered mobile web apps) $+$ 10 (de-cluttered mobile web apps)) $x$ 15 (trial repetitions) $=$ 900 measurements.

\subsection{Data Analysis}


% exploration
First of all, the resulting energy consumption data is visualized and explored using multiple \textcolor{blue}{scatter and box plots} to visually analyze the distribution of the data.
% check for normality
Afterward, we check for normality using a Q-Q plot and the Shapiro-Wilk test with $\alpha = 0.05$ to decide which statistical test can be applied to test our hypothesis.
% hypothesis testing
\textcolor{blue}{Given the results provided by the two previous steps, we apply a paired t-test if the data is normally distributed. The paired t-test is done using the energy consumption measurements as data and the two treatments (cluttered and de-cluttered) as the two populations that have to be compared. In the case that the data is not normally distributed, we would use the Wilcoxon signed-rank test instead of the paired t-test. Following that we have three blocks, we split the data into three parts and execute the paired t-test per page size block: small, medium, and large. We then use the Bonferroni correction to keep the total p-value at 0.05. The effect size is the quantitative measure of the strength of a phenomenon. In our study, this refers to the actual difference in energy consumption if the cluttered and de-cluttered energy consumption differed significantly. As this is not the case in our study, the effect size is irrelevant. }



%\textcolor{red}{Page limit: 3} 
% Report about: how you plan to conduct your experiment, which tools you are going to use, which devices/laptops, figure and description of the overall software/hardware infrastructure you are setting up for the experiment (\eg who communicates with whom, proxies, network requests, order of actions, \etc).
%\textcolor{red}{Page limit: 2}

\section{Experiment Execution}\label{sec:experumnent_executuon}

In our experiment, we use an Android smartphone to load the mobile web apps. The smartphone used for the experiment is the \textcolor{blue}{Google Pixel 3 and has an octa-core (4x2.5 GHz Kryo 385 Gold and 4x1.6 GHz Kryo 385 Silver) processor with 4 GB of RAM}. This smartphone runs Android version 8.0.0. To mitigate the effects of background processes, we disable services we do not use such as Bluetooth, Location Services (GPS), and NFC. To ensure that the experiment stays running while minimizing the impact of the display on the measured energy consumption, the screen brightness is minimized while ensuring that the phone cannot go to sleep mode. Furthermore, we disabled all push notifications and non-critical apps. 

\begin{figure}[h]
\centering
\includegraphics[width=9cm]{reportTemplate/figures/execution.pdf}
\caption{\textcolor{blue}{Experiment Execution}} \label{fig:deploymentview}
\end{figure}

The smartphone is connected to a laptop using Android Debug Bridge (adb)\footnote{\url{https://developer.android.com/studio/command-line/adb}}. The computer that we use to manage the experiment has an Intel i7 CPU (8th generation, 8 cores, and 4.2 GHz maximum clock speed) with 16 GB of RAM. Adb is a tool that allows communication between a development machine and an Android device. This communication entails information of the experiment configurations from the laptop to the smartphone and measurement data from the smartphone to the laptop. The Android smartphone and laptop are both connected to the same LAN. All the devices are in close proximity \textcolor{blue}{(less than a meter)} to the WiFi router (802.11ac standard and 5 GHz frequency band) to ensure minimum latency. The Android smartphone uses the Google Chrome web browser version \textcolor{blue}{{66.0.3359.158}} to access the internet and load the cluttered and de-cluttered mobile web apps. The researchers of the JSCleaner paper~\cite{chaqfeh2020jscleaner} cached the cluttered and de-cluttered mobile web apps, which they used for their research, on a proxy server and shared this proxy with us. In order to access the proxy server, the mitmproxy\footnote{\url{https://docs.mitmproxy.org/}} certificate \textcolor{blue}{is} installed on the smartphone. Mitmproxy is an HTTPS proxy that allows (among other features) interactive HTTPS requests. 

In order to \textcolor{blue}{execute} the experiment, we use Android Runner (AR)\footnote{\url{https://github.com/S2-group/android-runner}}. AR is a tool that facilitates the execution of measurement-based experiments on native apps and mobile web apps running on Android devices \cite{malavolta2020runner}. AR allows us to fill in the experiment configurations and setup after which it communicates this to the smartphone. After the experiment is conducted, AR stores the recorded measurement data on the laptop. AR utilizes Monkey Runner\footnote{\url{https://developer.android.com/studio/test/monkeyrunner}} to implement an API that grants control over the Android smartphone. We use Monkey Runner to construct a log of smartphone input commands. This log can be used to automate the interactions with the smartphone. \textcolor{blue}{Lastly, we use the Batterystats\footnote{\url{https://github.com/S2-group/android-runner/tree/master/AndroidRunner/Plugins/batterystats}} utility to measure the energy consumption. We chose this profiler as it was compatible with the used smartphone. Furthermore, Tan et al.~\cite{integrationtan} found an indication of Batterystats providing more accurate results of the power consumption compared to Trepn\footnote{\url{https://github.com/S2-group/android-runner/tree/master/AndroidRunner/Plugins/trepn}}. Batterystats is a plugin of AR.}

A high-level view of our experiment execution is visualized in Figure \ref{fig:deploymentview}. \textcolor{blue}{The experiment includes three main components: the \textit{Laptop}, \textit{Android Smartphone}, and \textit{MITM Proxy}. The experiment starts with sending commands from the \textit{Laptop} to the \textit{Android Smartphone} (step \textcolor{red}{1}). The connection between the \textit{Laptop} and the \textit{Android Smartphone} is established using \textit{Android Debug Bridge}. \textit{Monkey Runner} handles the automation of the external control of the smartphone (e.g. Terms and Conditions in browser app) and \textit{Batterystats} takes care of measuring the power consumption. These processes are orchestrated and managed by \textit{Android Runner}. Once the \textit{Android Smartphone} receives the commands, it requests (step \textcolor{red}{2}) the relevant \textit{Cluttered or De-Cluttered Mobile Web Apps} from the \textit{MITM Proxy}. The \textit{MITM Proxy} sends (step \textcolor{red}{3}), after checking the \textit{MITM Certificate}, the \textit{Cluttered or De-Cluttered Mobile Web Apps} to the \textit{Android Smartphone}. Lastly, the log data and measurement results are collected by \textit{Android Runner} (step \textcolor{red}{4}).}

After each run, the Google Chrome browser is cleared and closed. Then, the smartphone remains idle for \textcolor{blue}{2} minutes to ensure that all processes are finished. To guarantee the intrinsic variability of the experiment, each trial is repeated 15 times. We repeat this measurement for 10 randomly selected cluttered mobile web apps and their de-cluttered version. 
\section{Results}
Provide:
\begin{itemize}
\item descriptive statistics
\item hypothesis testing
\end{itemize}
Provide suitable plots and tables to illustrate your results.

\textcolor{red}{Page limit: Open - go deep as you wish}
\section{Discussion}

% Elaborate on the obtained results

In this section, we elaborate on the obtained results and state the implications of these results.
% intrinsic variability
First of all, we studied the variance level of the repeated trials. A trial is a combination of a subject (mobile web app) and a treatment (cluttered or de-cluttered). We measured the energy consumption of each trial 15 times to understand the external effects on the measurements. From these results, we learned that most trials were relatively stable (the outputs of the different runs had a low SD). However, few trials did result in a relatively high (between 2 and 6) SD. This can be due to fluctuations in the network quality, internal processes from the phone interfering with the experiment, or measurement errors. The variance in the measurements harms the reliability.

% data distribution
Second, we investigated the data distribution. We observed that the energy consumption of the cluttered and de-cluttered mobile web apps was relatively similar. Furthermore, we compared the energy consumption for the small, medium, and large mobile web apps. It is striking that the mean energy consumption of the mobile web apps of a medium page size have a higher mean compared to the mobile web apps of a large size. This could be due to the relatively small sample size. % testing normality 
Next, we tested the normality, from both the Shapiro-Wilk test and the Q-Q plot we concluded that the data is normally distributed.

% paired t-tests
The main research question of this research is ``\textit{What is the impact of cluttered JavaScript code in mobile web apps with respect to the energy consumption?}''. The null hypothesis states that the mean energy consumption of cluttered mobile web apps is equal to the mean energy consumption of the de-cluttered mobile web apps. In other words, we aim to find out whether cluttered JS code impacts the energy consumption. Using the results presented in Section~\ref{sec:results}, we are unable to reject the null hypothesis as the executed paired t-tests did not find p-values lower than the described values for $\alpha$. The research question is thus answered by claiming that de-cluttering does not impact the energy efficiency of mobile web apps. This claim holds for small, medium, and large mobile web apps.

% Impact into practice (industry) & Make results actionable
Our results show that cluttered mobile web apps do not influence the energy consumption. In the study presenting JSCleaner \cite{chaqfeh2020jscleaner}, the researchers did find that de-cluttering mobile web apps using JSCleaner reduces the PLT by 30\% and the number of requests and page size by 50\%. It thus seems that the choice of using a de-cluttering tool like JSCleaner depends on the goal. If the goal is reducing the PLT, it is a good decision to apply a de-cluttering tool. However, if the goal is to reduce the energy consumption, developers should not (with the current results in mind) apply such a de-cluttering tool.

% Lessons learnt
Hence, the lessons learned of our study are that applying JSCleaner has no impact on the energy consumption. However, we do want to emphasize that more research is required to support this claim and generalize this for all de-cluttering methods. A major implication on the validity of the results, and thus the conclusion, is that a relatively small sample size is used and solely one de-cluttering method is applied. 


% \textcolor{red}{Page limit: 1}
\section{Threats To Validity}\label{sec:threats}

In this section, we discuss the threats to validity and mitigation measures. Validity is defined as the extent to which results are sound and applicable to the real world \cite{wohlin2012experimentation}. The used validity classification is proposed by Cook and Campbell \cite{cook1979quasi}.

\subsection{Internal Validity}

Internal validity considers the causality between treatment and outcome \cite{wohlin2012experimentation}. 

\noindent \textbf{History}
To mitigate the effects of history, we ensured that both the cluttered and de-cluttered mobile web apps were cached at the same time. This ensures that the subjects did not change and can be properly compared. Next, the experiment trials are performed within the same time-frame. If this would not be a single type of time-frame, this could harm the validity as executing the experiment in a different time-frame could lead to different results. We mitigated this threat by performing the experiment on a regular day and night (Wednesday to Thursday and no special holiday). This ensured a representative influence of e.g. data-traffic and connectivity. Furthermore, we chose this setup to ensure that the experiment setting is similar in each trial. 

\noindent \textbf{Maturation} 
To mitigate the effect of maturation, we ensured that after each run possible effects on the smartphone and experiment setup of the previous run are eliminated. This entails that we removed the caches of the web browser and paused the smartphone for 2 minutes after each run to ensure that activated background processes are terminated. 

\noindent \textbf{Selection}
A few mobile web apps were not available. To ensure that this is not some specific type of subject we checked the mobile web pages that were unavailable manually to understand the issue. The reason was that some mobile web apps did not properly load. This could be due to network issues or internal issues of the proxy server. The omitted mobile web apps were of different topics. However, we cannot be completely certain that this is not a specific type of subject. 

\noindent  \textbf{Reliability of Measures}
In order to reduce the effects of co-factors and noise, we ensured that the experiment execution was similar for each experiment run. For instance, we requested the mobile web pages from the same proxy, through the same LAN, by the same device, using the same settings (e.g. browser, profiler). Moreover, we used the page size as a blocking factor to mitigate its effect on the results. However, not all experiment factors could be controlled. A threat to the internal validity is the fact that we used an external proxy to retrieve the subjects. Due to time constraints, we did not de-clutter the mobile web apps ourselves. This caused a level of uncertainty and lack of control over the web requests. Moreover, factors such as network instability and background processes within the device were out of our control. We mitigated these effects by turning off irrelevant applications, notifications, network and location services. Furthermore, the phone settings that could impact the energy consumption such as screen brightness remained constant throughout the experiment. To ensure the correctness of our measurements, we used a well-tested framework and profiler to manage our experiment. Finally, to guarantee the reliability of our measurements, we repeated each trial multiple times. The outcomes of the trial are relatively stable (most SDs are between 0 and 1). However, there are outliers with a higher SD. This means that there are other processes interfering with the measurements.

\subsection{External Validity}

External validity concerns the generalizability of the experiment \cite{wohlin2012experimentation}.

\noindent \textbf{Interaction Selection and Treatment}
To guarantee the generalizability of the results, we selected diverse, real-world mobile web apps. The mobile web apps originate from The Majestic Million\footnote{\url{https://majestic.com/reports/majestic-million}}. This selection contains operational mobile web apps from a wide variety of themes such as e-commerce, news, and sports. From this selection, we generated a random sample of subjects. A threat to the external validity is the relatively small sample size. The random selection mitigates this threat. 

\noindent \textbf{Interaction Setting and Treatment}
Another threat to the external validity is the controlled research setting. In the real-world, background processes such as push-notifications and location services are enabled. Hence, our experiment setting is not representative of a real-life setting. Furthermore, by ensuring that the research setup remains constant (e.g. using the same device and network connection), we threaten the external validity as, for instance, other devices and network conditions would perform differently. This is an inevitable trade-off between external vs. internal validity. We mitigated this effect by using a new device (Google Pixel 3) in a regular household setting.

\noindent \textbf{Interaction History and Treatment}
To mitigate the effects of the interaction between history and treatment, we performed the experiment on a regular work-day. As the experiment took a relatively long time to execute, we did run it partly over night. This is a potential threat to the validity.

\subsection{Construct Validity}

Construct validity concerns the relation between theory and observation \cite{wohlin2012experimentation}. 

\noindent \textbf{Inadequate Preoperational Explicitation of Constructs}
To ensure that the constructs are sufficiently defined, we applied the GQM method \textit{a priori}. This method allowed us to accurately define the goal, research question, and metrics.

\noindent \textbf{Mono-Operation Bias}
To ensure that we properly measured the effect of cluttered JS code on energy consumption, we measured the energy consumption of both the cluttered and de-cluttered version of the mobile web apps. Furthermore, we executed the experiment on different subjects and repeated each trial 15 times. However, the fact that the JS code is solely de-cluttered by one de-cluttering algorithm (i.e. JSCleaner) harms the validity of the experiment. Further research is required to test the effect of cluttered JS code on energy consumption using other de-cluttering algorithms. 

\noindent \textbf{Mono-Method Bias}
A threat to the construct validity is the fact that we used one metric (i.e. power consumption) to measure the energy consumption. This effect is mitigated by considering that it is an objective metric rather than a subjective metric.

\subsection{Conclusion Validity}

Conclusion validity concerns whether the relationship between the treatment and outcome is statistically correct and significant \cite{wohlin2012experimentation}.

\noindent \textbf{Low Statistical Power}
A threat to conclusion validity is our relatively small number of subjects. Due to time constraints, we tested 9 subjects within each category. The outcome of the statistical tests may be erroneous due to the small sample size. To mitigate this effect, future experiments are advised to test a larger sample. 

\noindent \textbf{Violated Assumptions of Statistical Tests}
A threat to the validity is that normality tests typically have a low power in small sample sizes. Considering our small sample size, it could be that the normality assumption is unjust. We mitigated this effect by detecting normality using two methods, namely, the Shapiro-Wilk test and the Q-Q plot. 

\noindent \textbf{Fishing and The Error Rate}
In our experiment, we conducted multiple statistical tests. Namely, for mobile web apps with a small, medium, and large page size. To mitigate the error rate of conducting multiple statistical tests, we adjusted the p-values accordingly.

%\textcolor{red}{Page limit: 1}
\section{Conclusions}\label{sec:conclusions}

% One brief paragraph for summarizing the main findings of the report.
In this study, we analyzed the energy consumption of cluttered JS code in mobile web apps. To do so, we randomly selected 27 subjects from the majestic million set. The subjects were de-cluttered using JSCleaner \cite{chaqfeh2020jscleaner}. We utilized Android Runner \cite{malavolta2020runner} to orchestrate and manage the experiment, and the Batterystats profiler to measure the energy consumption. Furthermore, we used statistical tests to conclude whether there is a significant difference in the energy consumption of the cluttered and de-cluttered mobile web apps. From these tests, no significant difference was found between the energy consumption of the cluttered and de-cluttered mobile web apps. This implies that developers do not need to take the energy consumption into account when applying a de-cluttering engine to their code. Considering that we used a small sample size to perform our experiments and tested solely one de-cluttering algorithm (JSCleaner), future studies are required to support this claim. %

% One brief paragraph about the possible extensions of the performed experiment (imagine that other 3 teams will be assigned to the extension of your experiment).
Future work could analyze a more extensive set of mobile web apps. Furthermore, the experiments could be performed on more devices (e.g. low- and high-end), operating systems (e.g. both iOS and Android), and browsers (e.g. Chrome, Firefox, Safari). Experiments done using these additions could substantiate the generalizability of the results and conclusions. In addition, more de-cluttering methods should be applied to the mobile web apps to test the difference between the effectiveness of the de-cluttering methods.

% Possible Extensions:
% 1. More samples
% 2. Real world setting
% 3. More devices
% 4. Apply more/other Cleaning methods
 

\bibliographystyle{IEEEtran}
\bibliography{references}

\end{document}
